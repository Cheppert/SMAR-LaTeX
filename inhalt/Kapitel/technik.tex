\chapter{Technik}
\label{cha:technik}
Das folgende Kapitel beschäftigt sich mit der bei diesem Projekt einzusetzenden Technik. Wie der Titel dieser Arbeit bereits verdeutlicht, soll dieses Projekt mit Hilfe von Wearable-Computern umgesetzt werden. In den folgenden Unterkapiteln sollen darüber hinaus weitere Gerätetypen dargestellt und miteinander verglichen werden. Die benutzte Vuzix M100 Smartglasses soll bewertet werden. Beim Vergleich der Gerätetypen wird nur die Hardware verglichen, die Software wird später definiert und im Rahmen dieses Projektes entwickelt. Dieses Kapitel soll sicherstellen, dass die eingesetzte Technik die Aufgaben am Besten unterstützt.\\

Beim Vergleich der Hardware wird insbesondere auf die definierten Ziele\footnote{Siehe Kapitel \ref{sec:zieldefinition} \nameref{sec:zieldefinition}} eingegangen, die mit Hilfe dieses Projektes in Verbindung mit der hier ausgewählten Hardware erreicht werden sollen. Außerdem wird hauptsächlich auch auf die Usability der Geräte eingegangen, die Voraussetzung für eine effektive und effiziente Arbeit mit dem Produkt ist. 

\section{Gerätetypen}
\label{sec:geraete}
In Kapitel \ref{cha:bedarfsanalyse} \nameref{cha:bedarfsanalyse} wurde der Bedarf an elektronischer Hilfe bei der Warenannahme/-einräumung erläutert, um auf dem Markt langfristig wettbewerbsfähig zu bleiben. Die Warenannahme und -einräumung findet an verschiedenen Stellen der Filiale statt und das sowohl im Lager- als auch im Kundenbereich. Feststehende große Gerätschaften sind somit nicht praktikabel und werden bereits ausgeschlossen. Voraussetzung sind somit mobile Gerätetypen mit aktuellen Schnittstellen \bzw Verbindungsmöglichkeiten, die eine reibungslose Integration in jedes Firmennetzwerk ermöglichen.\\
Aus diesen Anforderungen lassen sich folgende Gerätetypen ableiten:
\begin{itemize}
	\item Laptop
	\item Tablet-PC
	\item Tablet
	\item Smartphone
	\item Wearable Computer
\end{itemize}
Laptops bieten sehr viel Rechenleistung bei einem - gegenüber klassischen Tower-Computern - vergleichsweise geringem Gewicht. Gegenüber den anderen Gerätetypen sind diese bei weitem leistungsstärker und bieten sehr viele Anschlussmöglichkeiten ohne an die Leistungsgrenzen zu stoßen. Diese Rechenleistung ist in diesem Projekt aber nicht erforderlich, da die Anwendung keine großartigen \ac{3D} -Grafikberechnungen durchführen muss und keine aufwändige mathematische Formeln gelöst werden müssen. Ein Laptop ist jedoch deutlich unhandlicher als die anderen Gerätetypen und benötigt durch die Eingabegeräte (Tastatur und Maus) mehr Aufmerksamkeit des Benutzers. Augmented Reality lässt sich durch einen Laptop außerdem nicht umsetzen, da die Nutzung beide Hände benötigt - ein Einräumen der Ware und gleichzeitiges Benutzen des Laptops, der Hinweise auf den Warenstandort gibt, ist nicht möglich.\\

Tablet-PCs\footnote{Tablet-PCs werden in dieser Arbeit als Computer mit Touchbildschirm definiert, die durch bestimmte Technik schnell in ein Tablet ähnliches Gerät durch den Benutzer verwandelt werden können.} \bzw Tablets besitzen nicht soviel Leistung 

TODO: Vergleich

\section{Vuzix M100}
\label{sec:vuzix}
Eine Betrachtung der verschiedenen Möglichkeiten zeigt, wie bereits beschrieben, dass eine Datenbrille besondere Vorteile aufweist. Für die Betrachtung und Bedienung des Gerätes ist keine freie Hand nötig, sodass das Einräumen in ein Regal nicht behindert wird.\\

Für das in dieser Studienarbeit behandelte Projekt \glqq Shelf-Management mit Hilfe von augmented Reality\grqq\ stand die \glqq Vuzix M100 Smart Glasses\grqq\ Datenbrille zur Verfügung.

\subsection{Technische Daten}
Die Datenbrille von Vuzix ist mit einem 1 \ac{GB} großen LPDDR2 400 \ac{MHz} Arbeitsspeicher und einem OMAP4430 Prozessor ausgestattet, der Dual-Core Prozessor basiert auf der \ac{ARM} Architektur und taktet mit bis zu 1 \ac{GHz}.\footnote{\citep{omap4430}} Die 4 Gigabyte Flash-Speicher können durch den Micro-\acs{SD} Kartenslot erweitert werden, der eine Kartengröße bis 32 \ac{GB} unterstützt. Betrieben wird dieses System mit Android Ice Cream Sandwich (Version 4.0).\footnote{\citep{vuzixm100}}\\

Das Display hat eine Auflösung von 432*240 Pixel (Breite*Höhe) bei einem Breitbild-Seitenverhältnis von 16:9.\footnote{\citep{wqvga}} Außerdem bietet es eine Farbtiefe von 24 bit. Durch die Wölbung des Displays um 15 Grad, wirkt das Display der getragenen Brille – laut Hersteller - wie ein Monitor mit einer Bilddiagonale von 4 Zoll.\footnote{circa 10 \ac{cm}}\\

Die 5 Megapixel Kamera nimmt Bilder und Videos ebenfalls im 16:9 Seitenverhältnis auf. Videos werden in Full-HD Auflösung aufgenommen (1920*1080 Pixel). Allerdings besitzt die Kamera weder einen Blitz noch ein LED-Licht zur Verbesserung von schlechten Lichtverhältnissen.\\

Das Gerät kann über verschiedenste Wege bedient werden. Auf der Seite befinden sich 4 Kontrollknöpfe, die zum Ein- und Ausschalten, sowie für das Bewegen durch und Selektieren im Menü benutzt werden können. Das Mikrofon ermöglicht die Kontrolle per Sprache, entsprechende \glqq Nuance Voice Control\grqq -Software wird über das Betriebssystem geliefert. Das \glqq Noise Cancelling Microphone\grqq\ nimmt Umgebungsgeräusche auf und filtert sie aus dem Eingang des Spracherkennungs-Mikrofon raus. Dadurch ist die Stimme auch in lauten Umgebungen deutlicher und die Spracherkennung/-kontrolle funktioniert auch dann. Außerdem ist es möglich das Gerät über Gesten (wie z.B. Nicken, Kopf nach links/rechts schwenken) durch die eingebaute \glqq 3 DOF Gesture Engine\grqq\ zu steuern.\footnote{\citep{vuzixm100}}\\

Die Verbindung zwischen der Vuzix M100 Datenbrille und anderen Geräten, Netzwerken kann drahtlos über WLAN im Standard 802.11b/g/n oder über Bluetooth 4.0 hergestellt werden. Drahtgebunden kann das Gerät über USB verbunden werden. \ac{USB} ist ebenfalls die Schnittstelle über die neue Softwareupdates eingespielt werden können und das Gerät aufgeladen werden kann.\footnote{\citep{vuzixm100}}\\
Im \ac{WLAN} erreicht das Gerät Datenübertragungsraten von maximal 150Mbit/s, dies entspricht dem IEEE802.11n Standard auf dem 2.4GHz Frequenzband.\footnote{\citep{uebertragungsgeschwindigkeit}} Dies reicht für das Aufrufen von Webseiten, Übertragen von Bildern mit einer Auflösung von 432*240 Pixel (Breite*Höhe), sowie für das Übertragen weiterer Informationsdaten zur Bestimmung der Position einer Ware binnen weniger Sekunden aus.\\
Da Bluetooth 4.0 eine maximale Reichweite von maximal 100 Metern erreichen kann (und das auch nur theoretisch, gängig sind Reichweiten von ca. 10 - 50 Metern)\footnote{\citep{bluetooth}} und eine Übertragungsrate von nur ca. 1-2 Mbit/s erreicht, eignet es sich nicht zur Datenübertragung von Regalbildern oder komplexen Positionsbeschreibungen und Datenbankabfragen nach einem Produkt. Bluetooth eignet sich hingegen gut für die Verbindung zu Host-Systemen zur externen Steuerung der Datenbrille\footnote{Voraussetzung ist ein Android-Handy mit entsprechender App} und zur Anbindung externer Geräte, wie z.B. einem Bluetooth Barcode-Scanner.

\subsection{Bewertung}

TODO: Verbesserungswürdig, wird im Kapitel Ausblick näher beschrieben
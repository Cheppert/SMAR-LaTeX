\chapter{Rechteverwaltung - Web}
\label{cha:rechteverwaltung_web}
Die Web Administration stellt die zentrale Kontrolleinheit des gesamten Projekts dar. Über die Web Administration werden sämtliche Einträge der Datenbank verwaltet, dies schließt neben der Benutzer- und Geräteverwaltung ebenfalls die Verwaltung aller Regale und Produkte ein.\\
Dies sollte selbstverständlich ausschließlich durch autorisierte Personen durchführbar sein.\\

Die folgenden Kapitel befassen sich mit der Identifikation (\acs{AuthN}) und mit der Berechtigungskontrolle (\acs{AuthZ}) eines Benutzers gegenüber dem Webserver.

\section{\acf{AuthN}}
Ruft ein Benutzer eine URL der Webadministration auf, so wird zunächst überprüft, ob bereits eine mit Inhalt gefüllte \ac{PHP}-Session besteht, ist dies nicht der Fall wird der Benutzer auf die Login-Seite weitergeleitet. Der Benutzer hat nicht die Möglichkeit ohne \acl{AuthN} auf die Startseite oder eine Unterseite der Anwendung zu gelangen. Dementsprechend können ebenfalls keine Funktionen aufgerufen werden.\\
Ein Zugriff auf die \ac{REST} \ac{API} ist ohne Anmeldung ebenfalls nicht möglich, da der \ac{JWT} erst bei der Anmeldung generiert wird und Anfragen ohne gültigen \ac{JWT} abgebrochen werden.\\

Die Login-Seite hält ein \ac{HTML}-Formular mit zwei Eingabe-Feldern bereit, die die Eingabe des Benutzernamens und des Passworts ermöglichen. Diese Daten werden durch Absenden des Formulars an ein \ac{PHP}-Skript auf dem Server verschickt. Dieses Skript ruft den Eintrag der Datenbank ab, bei dem der Benutzername mit dem eingegebenen Namen identisch ist. Dem eingegebenen Passwort wird anschließend der Salt-Wert, der dem Benutzer in der Datenbank zugeordnet ist, angehängt und das zusammengesetzte Passwort wird mit dem SHA256-Verfahren gehasht. Das Passwort in der Datenbank wurde ebenfalls mit dem selben Verfahren gehasht und sollte daher identisch mit dem gehashten eingegebenen Passwort sein. Hat die Anfrage nach dem Benutzernamen einen Eintrag zurückgeliefert und die gehashten Passwörter sind identisch, so war der Login erfolgreich.\\
Bei einem erfolgreichen Login werden anschließend folgende Daten in einer neu erstellten \ac{PHP}-Session gespeichert:
\begin{itemize}
	\item Benutzer-ID (Primary Key der Datenbank)
	\item Benutzername
	\item Vorname
	\item Nachname
	\item gehashtes Password
	\item Personalnummer
	\item Berechtigungsstufe
	\item Loginzeit (Datum + Uhrzeit)
	\item Zeit seit dem letzten Seitenaufruf (Datum + Uhrzeit)
	\item ein gültiger \ac{JWT} zur Authentifizierung gegenüber der \ac{REST} \ac{API}\footnote{Siehe Kapitel \ref{cha:jwt} \nameref{cha:jwt}}
\end{itemize}
Der \ac{JWT} wird im Rahmen der der Session-Erstellung generiert.\\
Nach Generierung der Session wird nun die Startseite der Anwendung angezeigt.\\
Sollte der Login-Vorgang aufgrund einer ungültigen Benutzername/Passwort-Kombination so wird die Login-Seite mit einer entsprechenden Fehlermeldung erneut angezeigt.\\

Das Hash-Verfahren für das Passwort, so wie ein individueller Salt-Wert pro Benutzer stellen eine Authentifizierung nach aktuellem Standard mit hoher Sicherheit dar. Auch wenn einem Angreifer das Auslesen der Benutzerdaten aus der Datenbank gelingt, kann er das Passwort eines Benutzers und somit den Zugriff mit einer vorgefertigten Rainbowtabelle nicht erlangen. Er muss eine große, für jeden Benutzer individuelle Rainbowtabelle anlegen, die bei einer Passwortlänge von mindestens 64 Zeichen (Salt-Wert-Länge) bei mindestens 52 möglichen Zeichen (A-Z und a-z) mehrere hundert Terabyte groß ist.\footnote{TODO: Rechnung}\\

TODO:
\begin{itemize}
	\item Session-Überprüfung/Aufrufen weiterer Seiten bei Anmeldung
	\item \ac{REST} \ac{API}
	\item Benutzerverwaltung erklären
\end{itemize}

\section{\acf{AuthZ}}
Im Gegensatz zu den Benutzern der \ac{VR}-Geräte, die entweder durch ihre Aufgabe volle Berechtigung auf den \ac{AR}-Geräten oder keine Berechtigung benötigen und es somit keine verschiedene Berechtigungsstufen benötigt, gibt es in der Web Administration verschiedenste Benutzern mit unterschiedlichen Berechtigungen im Unternehmen.\\
Während der Filialleiter sowohl Zugriff auf die Benutzerverwaltung, als auch auf die Regal- und Produktverwaltung haben sollte, so sollte ein Mitarbeiter, der für die Warenannahme und -einräumung beauftragt wurde, keinen Zugriff auf die Benutzerverwaltung haben.\\
Vor Allem durch die Benutzerverwaltung ist hier ebenfalls auf rechtliche Bestimmungen zu achten. § 3a Satz 2 \acs{BDSG}\footnote{\acf{BDSG}}:
\begin{quote}
	\glqq \textit{Die Erhebung, Verarbeitung und Nutzung personenbezogener Daten und die Auswahl und Gestaltung von Datenverarbeitungssystemen sind an dem Ziel auszurichten, so wenig personenbezogene Daten wie möglich zu erheben, zu verarbeiten oder zu nutzen. Insbesondere sind personenbezogene Daten zu anonymisieren oder zu pseudonymisieren, soweit dies nach dem Verwendungszweck möglich ist und keinen im Verhältnis zu dem angestrebten Schutzzweck unverhältnismäßigen Aufwand erfordert.}\grqq
\end{quote}
sagt aus, dass auch die Nutzung von personenbezogenen Daten, wie sie in der Benutzerverwaltung abgespeichert werden müssen, so weit wie möglich reduziert werden soll. Ein Mitarbeiter, der nicht im Personalmanagement angestellt oder mit Aufgaben der Benutzerverwaltung beauftragt ist, sollte somit auch keinen Zugriff auf personenbezogene Daten haben. Im Zweifelsfall wäre dieser Benutzer eventuell sogar unberechtigt diese Daten einzusehen.\\

Die Autorisierung von Benutzern konnte in \ac{SMAR} dahingehend vereinfacht werden, dass davon ausgegangen wurde, dass die verschiedenen Berechtigungsstufen aufeinander aufbauend sind. Das heißt jemand in einer höheren Berechtigungsstufe hat immer die Berechtigungen der darunterliegenden Berechtigungsstufe und darauf aufbauende Rechte. Eine niedrigere Berechtigungsstufe hat somit niemals Rechte, die eine höhere Stufe nicht besitzt.\\

Aus den Aufgaben dieses Projektes und der Berücksichtigung eventueller Gesetzesvorgaben ließen sich die folgenden acht Berechtigungsstufen herleiten:
%\begin{enumerate}
%	\item[0]: keine Berechtigung
%	\item[10]: Products, Units, Shelves, Sections lesen; keine Schreibberechtigung
%	\item[20]: Schreibberechtigung für Products \& Units
%	\item[30]: Schreibberechtigung für Bestellungen
%	\item[40]: Schreibberechtigung für Products, Units, Shelves \& Sections
%	\item[50]: Lese- und Schreibberechtigung für Geräteverwaltung
%	\item[60]: Lese- und Schreibberechtigung für die Benutzerverwaltung
%	\item[70]: Vollständige Berechtigung
%\end{enumerate}
\begin{figure}[H]
	\centering
	{\includegraphics[scale=0.5]{Bilder/role_web.jpg}}
	\caption{Berechtigungsstufen in der Web-Administration}
	\label{fig:role_web}
\end{figure}
Wie im vorherigen Absatz beschrieben, besitzen höhere Berechtigungsstufen automatisch auch die Berechtigungen geringerer Stufen. Außerdem wurde darauf geachtet, dass späteres Erweitern der Funktionalität der Anwendung noch weitere Berechtigungsstufen erfordern könnten. Berechtigungsstufen wurden in Zehnerschritten durchnummeriert, einzelne Berechtigungsstufen können durch Nutzen der Einerschritte in vorhandene Stufen integriert werden ohne dass die Anwendung in bestehenden Programmteilen angepasst werden muss.\\

Berechtigungsstufe 0 gibt dem Benutzer keine Berechtigung die Anwendung zu benutzen, der Anmeldebildschirm wird nicht auf die Startseite weitergeleitet, sondern gibt eine Fehlermeldung \glqq Insufficient Permissions\grqq\footnote{zu Deutsch: \glqq mangelnde Berechtigung\grqq} zurück. Dies kann erforderlich sein, wenn einem Mitarbeiter gekündigt wurde, er aber aufgrund von \zB offenen Gehaltszahlungen noch nicht aus dem System gelöscht werden darf. Berechtigungsstufe 10 gibt Zugriff auf die Anwendung und auf die Basisfunktionalität, der Benutzer bekommt allerdings noch keine Schreibberechtigung. Dies ist für Mitarbeiter sinnvoll, die mit der Überwachung des Warenbestands beauftragt wurden, allerdings keine Berechtigung haben sollen, diesen zu verändern. Die darauffolgende Stufe 20 gibt dem Benutzer zusätzlich die Berechtigung auf die Produktverwaltung. Der Benutzer ist daher berechtigt Produkte zu verändern/hinzuzufügen oder zu löschen. Die Regalverwaltung ist hier noch nicht inbegriffen und wird erst in der Stufe 40 hinzugefügt. Dem Benutzer ist es nun erlaubt Regale, Regalstandorte und die Anordnung der Produkte innerhalb des Regales zu verändern. Darauffolgende Berechtigungsstufen 50, 60 und 70 dienen der Verwaltung und sind für die eigentliche Aufgabenerfüllung nicht mehr notwendig. Stufe 50 fügt die Berechtigung zur Geräteverwaltung hinzu, das heißt der Benutzer darf nun \ac{AR}-Geräte, wie \zB die Smartglass hinzufügen oder löschen. Stufe 60 fügt eine nach dem Gesetz sensible Berechtigung hinzu, nämlich den Lese- und Schreibzugriff auf personenbezogene Daten. Der Benutzer ist nun berechtigt andere Benutzer hinzuzufügen, zu löschen oder zu verändern (wie \zB Berechtigungen verändern). Das Ändern des Passworts des eigenen, angemeldeten Benutzers ist jedoch bereits ab Berechtigungsstufen größer als 0 verfügbar. Die letzte Stufe 70 gewährt volle Berechtigungen auf alle Komponenten der Web-Administration. Dies ist für Systemadministratoren und Filialleiter geeignet, in diesen Fällen muss ein ständiger Zugriff auf die gesamte Anwendung garantiert sein.\\

Die oben genannten Berechtigungsstufen werden in der Datenbank in der für die Benutzer zuständigen Tabelle (user\footnote{Siehe Kapitel \ref{sec:architektur_datenbank} \nameref{sec:architektur_datenbank}}) abgespeichert. Die Tabelle enthält die Spalte role\_web. In dieser Spalte wird für jeden Benutzer die entsprechende Berechtigungsstufe als zweistellige Nummer gespeichert. Greift der Benutzer nun auf die Anwendung zu, wird während der Anmeldung zunächst überprüft, ob die Berechtigung ungleich 0 (keine Berechtigung) ist und anschließend die Berechtigung in der Session gespeichert. Ruft man eine Komponente/eine Funktion innerhalb der Anwendung auf, so wird überprüft, ob die Berechtigungsstufe größer oder gleich (>=) der erforderlichen Nummer ist. Ist die Nummer größer oder gleich wird der Zugang gewährt und die Funktion aufgerufen, ansonsten wird die Anfrage mit der Fehlermeldung \glqq Insufficient Permissions\grqq\footnote{\glqq mangelnde Berechtigung\grqq} abgebrochen.\\
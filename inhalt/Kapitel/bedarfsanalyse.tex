\chapter{Bedarfsanalyse}
\label{cha:bedarfsanalyse}
Dieser Abschnitt erläutert Grundlegendes zum Thema Shelf-Management, damit das Verständnis des Lesers ermöglicht wird und die Nachvollziehbarkeit gesteigert. Anfangs wird die Warenannahme, das Waren einräumen und der Kundenservice in einem Einzelhandel schematisch, beispielhaft und ohne technische Unterstützung dargestellt, analysiert und anschließend Problemstellen aufgezeigt. Dabei wird insbesondere auf die Probleme und Schwachstellen eingegangen.
\\
Zum Schluss werden aus den Problemen und Schwachstellen Ziele definiert, die im weiteren Verlauf der Arbeit weiter analysiert und schließlich erfüllt werden sollen.
\section{Warenannahme}
\label{cha:warenannahme}
Damit überhaupt Regale eingeräumt werden können, muss Ware in das Geschäft gelangen. Nach einer Bestellung durch geschultes Personal ist es üblich, dass Ware mit einem Transporter auf Paletten und einem beigelegten Lieferschein angeliefert wird. Sofern Ware bestellt und geliefert wurde, muss die Ware abgenommen werden. „Abnehmen“ bedeutet, in diesem Kontext, dass die Ware von der Filiale offiziell als geliefert gekennzeichnet wurde. Dazu gehört das Zählen der gelieferten Ware und dem Vergleich zur eigentlichen Bestellung. Beim Vergleich wird überprüft, ob die Menge der Bestellten Ware mit der gelieferten übereinstimmt. Dazu wird eine sogenannte Setzliste verwendet. Diese Setzliste ist nicht mit dem Lieferschein zu verwechseln. Die Setzliste ist eine Kopie der Bestellung und somit ein Mittel, um einerseits die Lieferung zu überprüfen und andererseits die Bestellung selbst. 
Wurde die Palette abgearbeitet und die Setzliste abgehackt, wird dieser zu einer Führungskraft des Geschäfts gebracht, damit dieser die Setzliste offiziell unterzeichnet und dann in ein elektronisches System überträgt. Nach diesem Arbeitsschritt ist es möglich in der Buchhaltung die entsprechenden Beträge zu verbuchen.
Solange die Ware nicht im Verkaufsraum benötigt wird, wird diese im Lager des Geschäfts aufbewahrt. 
Dieser Prozess ist in folgendem Aktivitätsdiagramm visualisiert und zeigt mögliche Fehlerquellen, Risiken und Problemstellen. Diese Erkenntnisse sind mithilfe von erfahrenen und sehr geschulten Führungskräften von einem führenden Einzelhändler in einer Umfrage erarbeitet worden. 
[Aktivitätsdiagramm einfügen]
Wie in der Grafik zu sehen, fängt der Prozess beim Bestellen von Ware an. Das verantwortliche Personal muss ohne technische Hilfsmittel selbst einschätzen können, wie viel Ware zum jeweiligen Zeitpunkt von diesem Typ Ware zu welchem Zeitpunkt bestellt werden soll. Dabei spielen verschiedene Aspekte eine große Rolle, wie etwa: der Wochentag, das Wetter, die Ferienzeit, etc. 
Folglich ist es nur wenigen, erfahrenen Mitarbeitern möglich dies durchzuführen. Es ist allerdings wünschenswert, dass möglichst viele Mitarbeiter schnell bestellen können, sodass man unabhängiger von bestimmten Personen vor Ort wird und gleichzeitig sorgfältige Bestellungen hat.
Der Schritt „Ware abnehmen“ erfordert die angesprochene Setzliste, also Papier. Dabei ist zu beachten, dass es vorteilhaft ist, wenn der Mitarbeiter die Hände frei zum Arbeiten an der Palette hat, um sie zu zählen. Dies ist spätestens, wenn er auf der Liste abhackt nicht mehr möglich. Gleichzeitig können Fehler beim Vergleichen der Liste geschehen, zum Beispiel, dass der Mitarbeiter sich verzählt. Zusätzlich sollte sofort bei Anlieferung die Ware überprüft werden, um die direkte Zuordnung zur Setzliste zu haben. So vermeidet man unnötigen Verlust oder Beschädigung der Liste und weitere Arbeit. 
Sofern es keine Setzliste mehr gibt, sondern durch eine elektronische Möglichkeit entsprechend ersetzt wird, hat der Mitarbeiter beide Hände immer frei zum Arbeiten. Da der Mitarbeiter, bei entsprechender Lösung, nicht mehr selbst vergleichen muss, ist die Wahrscheinlichkeit von Fehlern gesenkt und verspricht korrektere Buchungen sowie genauere Analysen im Anschluss. Außerdem ist es grundsätzlich nicht möglich eine elektronische Liste, oder Pendent, zu verlieren. 
Der nächste Arbeitsschritt, „Ware in Lager einräumen“, kostet im Normalfall kaum Zeit, auch bei Laien. Allerdings geht bei viel Ware viel Platz im Lager verloren, der im Idealfall anders genutzt werden könnte – nämlich zum Verkauf von weiterer  Ware. Platz ist eine kostbare Ressource, wie man zum Beispiel an Grundstückspreisen feststellen kann. Außerdem ist bei größerer Verkaufsfläche mehr Potential den Umsatz zu erhöhen. Bei angemessenen Bestellungen und korrekten Lieferungen wird Ware sofort auf die Verkaufsfläche gestellt, sodass sie, im besten Fall, gar nicht im Lager gelagert wird. 
Sobald eine materielle Setzliste abgelöst wurde, entfallen Arbeitsprozesse wie „Setzliste ins Büro bringen“ oder „Setzliste mit abgenommener Ware vergleichen“, sodass auch weiterer Mehraufwand durch verlorene oder falsche Lieferungen wegfällt. Dazu ist zu sagen, dass der Aufwand bei falscher Lieferung erstmal nur in der Filiale entfällt. Die übergeordnete Instanz, die sich um die filialspezifischen Buchungen kümmert, muss weiterhin damit arbeiten. Allerdings sind so die Daten schon von Anfang an digital und können so weiter verwendet werden ohne zusätzliches manuelles Eingreifen.

\section{Waren einräumen}
\label{waren_einräumen}
Der bis hierhin beschriebene Prozess bezieht sich ausschließlich darauf, die Ware vom Lieferanten entgegen zu nehmen und korrekt zu dokumentieren. Das angesprochene Shelf-Management selbst, beginnt erst ab diesem Zeitpunkt.
\\


Ein Mitarbeiter muss die Ware aus dem Lager heraus an die richtige Stelle in den Regalen einräumen, und dies möglichst zügig, da sonst, vor allem tiefgekühlte Ware, an Qualität verliert. Ein weiteres wichtiges Argument für schnelles Ware einräumen ist, dass sobald die Ware nicht im Regal steht, potentieller Umsatz verloren geht. Zusätzlich muss der Mitarbeiter mit der Ware, im Normallfall auf einer Palette oder vergleichbar großen Container, im Gang stehen und die Ware einräumen. Dabei steht er Kunden im weg, die sich oft darüber ärgern. Dies ist aus dem weiter oben angesprochenen Einzelhändler hervorgegangen. Vor allem, wenn der Kunde merkt, dass der Mitarbeiter nicht genau weiß, wo die Ware eingeräumt werden muss, verliert er das Vertrauen und die Bindung an den Mitarbeiter und somit an den Händler. Dabei ist Kundenzufriedenheit ein wichtiger Aspekt. Besonders ist der Punkt dann brisant, wenn der Kunde die Zufriedenheit mit dem Händler verliert. Daraus ergeben sich diese Folgen:
\\
\begin{itemize}
	\item Unzufriedenheit führt zur Abwanderung bisheriger Kunden.
	\item Unzufriedene Kunden betreiben negative Mundpropaganda und berichten durchschnittlich zehn bis zwölf weiteren Personen von ihrer Unzufriedenheit.
	\item Die Gewinnung von Neukunden verursacht gegenüber der Bindung einer Altkunden das Vier- bis Sechsfache an Kosten.
\end{itemize}

Diese Folgen beziehen sich auf den Dienstleistungssektor, der nicht 100\% auf den Markt des Einzelhandels trifft. Allerdings treffen die angesprochenen Punkte spätestens dann ein, sobald der Kunde einen Mitarbeiter fragt, wo ein entsprechender Artikel steht bzw. ob dieser vorhanden ist. Dann erfüllt der Mitarbeiter eine Dienstleistung an den Kunden, indem er ihm Auskunft gibt. Jedoch ist es sehr häufig der Fall, dass ein Mitarbeiter nicht weiß, ob oder wo ein Artikel vorhanden ist. Vor allem in großen Filialen ist dies der Fall.\\
Der wohl wichtigste Aspekt beim Ware einräumen ist, dass der Mitarbeiter dabei unterstützt wird. Vor allem bei großer Verkaufsfläche, vielen Regalen und vielen Artikeln ist es für, besonders unerfahrene, Mitarbeiter schwierig zu erkennen, wo genau der Artikel eingeräumt werden muss. Schwierig wird es selbst für erfahrene Mitarbeiter bei wechselnden Sortierungen.
\section{Kundenservice}
Aus dem Interview ergab sich auch das Problem, vor Allem bei großen Filialen, dass Mitarbeiter dem Kunden auf Anfrage nicht antworten können, ob ein Produkt noch vorhanden ist oder wo ein Produkt zu finden sei. Generell lässt sich genau durch dieses Wissen die zuvor angesprochene Kundenzufriedenheit erhöhen. Da diese Schwachstelle den zuvor angesprochenen sehr stark ähnelt, soll diese auch im weiteren Verlauf der Arbeit thematisiert werden. 
\section{Zusammenfassung der Schwachstellen}
Zusammengefasst sind folgende Punkte Schwachstellen, die gleichzeitig mehrere Auswirkungen haben:
\begin{itemize}
	\item Der Mitarbeiter muss erkennen, wann er welche Ware einräumen muss.
	\item Der Mitarbeiter muss wissen, wo er die Ware einzuräumen hat.
	\item Der Mitarbeiter muss wissen, wie viel von der Ware noch generell vorhanden ist.
	\item Der Kundenservice leidet durch mangelnde Kenntnisse über aktuelle Zustände in der Filiale
\end{itemize}
\section{Zieldefinition}
\label{sec:zieldefinition}
Aus den oben erwähnten Schwachstellen ergeben sich die Ziele, die in dieser Arbeit erreicht werden sollen. Konkret lassen sich folgende Ziele definieren:
\begin{itemize}
	\item Die Geschwindigkeit des Mitarbeiters, vor Allem bei ungeschulten Personal, soll beim Einräumen von Ware erhöht werden.
	\begin{itemize}
		\item Der Mitarbeiter soll technisch beim Einräumen der Ware unterstützt werden.
	\end{itemize}
	\item Der Mitarbeiter soll bei der Warenannahme entlastet werden.
	\begin{itemize}
		\item Bestellungen können verwaltet werden.
		\item Gelieferte Ware kann durch eine technische Lösung optimaler erfasst werden
	\end{itemize}
	\item Die Kundenzufriedenheit in der Filiale erhöhen
	\begin{itemize}
		\item Der Mitarbeiter soll in der Lage sein, dem Kunden den Platz eines bestimmten Artikels zu nennen
		\item Der Mitarbeiter soll in der Lage sein, dem Kunden den Vorrat eines Artikels nennen können. 
	\end{itemize}
\end{itemize}

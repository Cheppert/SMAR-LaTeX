\chapter{Implementierung der Webadministration}
\label{cha:impl_web}

Die Web Administration stellt die zentrale Kontrolleinheit des gesamten Projekts dar. Über die Web Administration werden sämtliche Einträge der Datenbank verwaltet, dies schließt neben der Benutzer- und Geräteverwaltung ebenfalls die Verwaltung aller Regale und Produkte ein.


\section{Technologische Grundlagen}

Die Administrationsoberfläche ist als Webapplikation und im Speziellen als Single-Page-Applikation konzipiert, d.h. sie besteht grundlegend aus einer einzigen Webseite, deren Layout (wie für Webseiten üblich) auf \ac{HTML} und \ac{CSS} aufgebaut ist. Die einzelnen Ansichten (Views) der Anwendung sowie jegliche dynamische Interaktionsprozesse werden über JavaScript (\acs{JS}) als clientseitige Scriptsprache gesteuert. Dabei werden auch Daten oder Views im Hintergrund asnychron über \ac{AJAX} nachgeladen. Durch dieses Grundkonzept werden viele Ladevorgänge der kompletten Webseite vermieden und nur die Daten nachgeladen, die im Einzelnen benötigt werden -- dies sorgt für eine bessere Performance der Anwendung und somit eine bessere User Experience.

Wie bereits in der Architektur angedeutet, läuft die Webadministration serverseitig mit der Scriptsprache \ac{PHP}. Hierüber werden Inhalte und Layoutkomponenten vorgeneriert und abhängig von den Parametern der clientseitigen Anfragen ausgeliefert.

Dennoch lassen sich Teile der Anwendung ohne Einsatz von JavaScript nur sehr aufwändig umsetzen; als Beispiel dafür sei hier der Shelf Designer genannt, der im Folgenden noch näher beschrieben wird.


\section{Aufbau der Arbeit}
\subsection{bla}
\subsubsection{blabla}

wie bereits in architektur angedeutet, Serverseitige Scriptsprache PHP (Implementierung)

kann theoretisch auf jedem gerät im netzwerk bereitsgestellt,m da nur anbindung an DB nötig - sinn aber, um geräte zu spraen, auf zentralen server
netzwerk entsprehcende größe: performancegründen eigener server

Slim Framework (Implementierung)

%TODO: Erläutern, warum Berechnung der Kapazität abgelehnt wird




Aufzählung:
\begin{itemize}
	\item basd
	\item ...
\end{itemize}





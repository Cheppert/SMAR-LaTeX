\chapter{Implementierung der Client-Schnittstelle}
\label{cha:impl_api}

Die Client-Schnittstelle dient der Kommunikation zwischen den Clients (z.B. Smartglass) und dem \acs{SMAR}-Server, um Zugriff auf die Daten aus der Datenbank zu erhalten und diese zu manipulieren.\\

Für diese Client-Schnittstelle fiel die Wahl auf eine \acs{REST} \acs{API}. Dabei handelt es sich um einen Webservice, der Ressourcen über fest definierte Routen (virtuelle Dateipfade auf dem Server) bereitstellt. Diese Routen können über die Standard-\acs{HTTP}-Befehle wie z.B. \emph{GET} oder \emph{POST} angesprochen werden und sind somit technologisch sehr flexibel -- \acs{HTTP}-Anfragen können von fast allen Programmiersprachen und -umgebungen versendet und empfangen werden.\\

Wie die Webadministration wurde die \acs{REST} \acs{API} über die Scriptsprache \acs{PHP} umgesetzt. Um den Arbeitsaufwand möglichst gering zu halten und gleichzeitig eine stabile API bereitstellen zu können, wurde das \emph{Slim Framework}(\url{http://www.slimframework.com/}) verwendet. Dabei handelt es sich um ein leichtgewichtiges Framework, mit dem Routen für die gängigen \acs{HTTP}-Befehle in \acs{PHP} programmiert werden.\\




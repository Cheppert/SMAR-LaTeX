\chapter{Vorwort}
\label{cha:vorwort}
Im deutschen Einzelhandel spielt das Regalmanagement (oder auch Shelf Management) eine entscheidende Rolle. Verschiedene Studien behandeln die \glqq On-Shelf Availability\grqq , die optimale Ressourcen- und Regalplanung und die richtigen \glqq Eye-Catcher\grqq an einem Regal.\\

Die Einzelhändler versuchen, die Produkte in Regalen immer in der richtigen Menge zur Verfügung zu stellen und einen \glqq Out-of-Stock\grqq -Zustand zu verhindern. Die richtige Planung soll dafür sorgen, dass Kunden die richtigen Produkte möglichst sofort finden und Regalplatz nicht unnötig verschwendet wird. \glqq Eye-Catcher\grqq sollen den Kunden außerdem zu vom Einzelhandel beworbenen Produkten leiten.\\

Ein Beispiel für diese Studien ist \zB die Studie \glqq Improving On-Shelf Availability - It Matters More\grqq (April 2012 von IRI).\footnote{\url{http://www.iriworldwide.eu/Portals/0/ArticlePdfs/OSA White Paper - Final.pdf}} Diese Studie setzt vor allem den Fokus auf die Kundenzufriedenheit.\\

Eine eigens im Januar 2015 durchgeführte Studie stellte ein weiteres Problem im Shelf Management heraus: Das Einsortieren und die Inventur durch die Mitarbeiter.
\begin{itemize}
	\item Wo soll die Ware genau einsortiert werden?
	\item Wie ist der Warenbestand im Lager und auf der Verkaufsfläche?
\end{itemize}
Aus diesen Fragen lassen sich weitere ableiten, deren Beantwortung für eine reibungslose Führung der Filiale unerlässlich ist.\\

Die folgende Studienarbeit wird sich mit diesen Fragen befassen, sie analysieren und beantworten. Der Fokus wird dabei hauptsächlich auf die Unterstützung der Mitarbeiter durch Technologien in Form von Wearable Computern und Augmented Reality gelegt.
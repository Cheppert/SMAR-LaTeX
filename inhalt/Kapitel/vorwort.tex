\chapter{Vorwort}
\label{cha:vorwort}
Im deutschen Einzelhandel spielt das Regalmanagement oder auch Shelf-Management eine entscheidende Rolle. Verschiedenste Studien handeln über die \glqq On-Shelf Availability\grqq , die optimale Ressourcen- und Regalplanung und über die richtigen \glqq Eye-Catcher\grqq an einem Regal.\\

Die Einzelhändler versuchen also Produkte in Regalen immer in der richtigen Menge zur Verfügung zu haben und ein \glqq Out-of-Stock\grqq -Zustand zu verhindern. Die richtige Planung soll dafür sorgen, dass Kunden die richtigen Produkte möglichst sofort finden und Regalplatz nicht unnötig verschwendet wird. \glqq Eye-Catcher\grqq sollen den Kunden außerdem zu vom Einzelhandel beworbenen Produkten leiten.\\

Ein Beispiel für diese Studien ist \zB die \glqq Improving On-Shelf Availability - It Matters More\grqq Studie aus April 2012 von IRI.\footnote{\url{http://www.iriworldwide.eu/Portals/0/ArticlePdfs/OSA White Paper - Final.pdf}}\\

Diese Studie hat vor Allem den Fokus auf der Kundenzufriedenheit.\\

Eine eigens durchgeführte Studie aus Januar 2015 stellte ein weiteres Problem im Shelf-Management heraus: Das Einsortieren und die Inventur durch die Mitarbeiter.
\begin{itemize}
	\item Wo soll die Ware genau einsortiert werden?
	\item Wie viel haben wir noch von der Ware?
\end{itemize}
Aus diesen Fragen lassen sich Weitere ableiten, dessen Beantwortung für eine reibungslose Führung der Filiale unerlässlich ist.

Die folgende Studienarbeit wird sich mit diesen Fragen befassen, analysieren und beantworten. Der Fokus wird dabei hauptsächlich auf die Unterstützung und Beantwortung der Fragen durch Technologie in Form von Wearable Computern und Augmented Reality gelegt.
\chapter{Vorwort}
\label{cha:vorwort}
Im deutschen Einzelhandel spielt das Regalmanagement\footnote{entspricht Shelf-Management} eine entscheidende Rolle. Verschiedene Studien behandeln die \glqq On-Shelf Availability\grqq ,~ die optimale Ressourcen- und Regalplanung und die richtigen \glqq Eye-Catcher\grqq\ an einem Regal. Ein Beispiel für diese Studien ist \zB die Studie \glqq Improving On-Shelf Availability -- It Matters More\grqq~ (April 2012 von IRI).\footnote{\url{http://www.iriworldwide.eu/Portals/0/ArticlePdfs/OSA White Paper - Final.pdf}}. Diese Studie setzt vor allem den Fokus auf die Kundenzufriedenheit.\\

Die Einzelhändler versuchen, die Produkte in Regalen immer in der richtigen Menge zur Verfügung zu stellen und einen \glqq Out-of-Stock\grqq -Zustand zu verhindern. Die richtige Planung soll dafür sorgen, dass Kunden die richtigen Produkte möglichst sofort finden und Regalplatz nicht unnötig verschwendet wird. \glqq Eye-Catcher\grqq~ sollen den Kunden außerdem zu vom Einzelhandel beworbenen Produkten leiten.\\

Eine eigens im Januar 2015 durchgeführte Studie stellte ein weiteres Problem im Shelf-Management heraus: Das Einsortieren und die Inventur durch die Mitarbeiter.
\begin{itemize}
	\item Wo soll die Ware genau einsortiert werden?
	\item Wie ist der Warenbestand im Lager und auf der Verkaufsfläche?
\end{itemize}
Aus diesen Fragen lassen sich weitere ableiten, deren Beantwortung für eine reibungslose Führung der Filiale unerlässlich ist.\\

Die folgende Studienarbeit wird sich mit diesen Fragen befassen, sie analysieren und beantworten. Um die Thematik an einem praktischem Beispiel zu behandeln, soll ein \glqq Proof of Concept\grqq~in Form eines Projektes entwickelt werden. Dieses Projekt heißt \ac{SMAR}. Im Rahmen dieses Projektes soll ein System entwickelt werden, welches das elektronische Shelf-Management ermöglicht. Der Fokus wird dabei hauptsächlich auf die Unterstützung der Mitarbeiter durch Technologien in Form von Wearable-Computern gelegt. Diese sollen ein noch relativ junges Konzept in der Informationstechnologie anwenden: Augmented Reality.

\begin{quote}
\glqq \textit{Augmented reality is a form of mixed reality where the live view of a real-world environment is enhanced by virtual (interactive) overlay techniques.}\grqq \footnote{\citep{augmented_reality}}
\end{quote}

Wie aus dieser Definition hervorgeht, ist Augmented Reality eine Anreicherung der natürlichen Sicht mit weiteren, computergenerierten Inhalten. Diese Inhalte können visueller oder informativer Art sein, \zB zusätzliche Informationen zur Umgebung oder Überlagerung von Videomaterial der Realität mit Bildern. Diese geben dem Nutzer einen Mehrwert, indem sie zusätzliche Informationen direkt anzeigen, ohne eine direkte Interaktion des Anwenders zu erfordern. Augmented Reality wird in der Regel mit Wearable-Computern verwendet.
\\
Dieses Prinzip soll in \acs{SMAR} also zusammen mit Wearable-Computern dazu genutzt werden, Mitarbeiter in einer Einzelhandelsfiliale bei der täglichen Arbeit so zu unterstützen, dass ihnen die Arbeit leichter fällt -- insbesondere bei der Warenannahme und beim Shelf-Management. 
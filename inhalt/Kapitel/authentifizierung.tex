\chapter{Rechteverwaltung - App auf Brille}
\label{cha:rechteverwaltung_vr}



\section{\acf{AuthN}}
Das folgende Kapitel beschäftigt sich mit der Authentifizierung in der App gegenüber dem Server. Benutzte Bibliotheken und Eingabemethoden werden erklärt. Darüber hinaus wird die verwendete Methode mit anderen technisch möglichen Eingabemethoden verglichen.\\

\acl{AuthN} dient der Identifikation einer Person/eines Gerätes.

\subsection{\acs{AuthN} gegenüber der Brille}
Die App auf der eingesetzten Vuzix M100 Virtual Reality Brille wird, wie beschrieben, zur Warenannahme, sowie zum Einräumen von Produkten verwendet - mit der Brille kann der Warenbestand daher aktiv verändert und manipuliert werden.\\
Diese Veränderung sollte, um \zB strukturierten Diebstahl zu vermeiden, nur durch Authentifizierte \acs{AuthN} und Autorisierte \acs{AuthZ} Personen durchgeführt werden.\\

Die gängige Ein-Faktor-Authentifizierung besteht aus der Kombination eines Benutzernamens mit einem Passwort. Dieses Verfahren hat sich bewährt und bietet bei korrekter Implementation eine durchschnittliche Sicherheit vor unbefugtem Zugriff. Diese Sicherheit würde im Rahmen dieser Anwendung ausreichen, da der Angreifer neben den Benutzerdaten, ebenfalls Zugriff auf ein Gerät haben muss, welches:
\begin{itemize}
	\item in das Firmennetzwerk eingebunden ist und
	\item gegenüber dem Server authentifiziert\footnote{s. Kapitel \ref{cha:authn_server}\nameref{cha:authn_server}} ist.
\end{itemize}
Eine Zwei-Faktor-Authentifizierung ist somit bereits gegeben, da der Benutzer sowohl Wissen (Benutzername und Passwort) als auch Besitz benötigt (Die authentifizierte \acl{VR}-Brille).\\
Die Grundlage für eine gute Anwendungssicherheit ist somit gegeben.

Die Brille hat, wie im Kapitel \ref{cha:brille}\nameref{cha:brille} beschrieben, folgende Eingabemethoden:
\begin{itemize}
	\item 4 Knöpfe an der Brille zur Navigation durch das Betriebssystem
	\item Sensoren zur Erkennung von Gesten
	\item Mikrofon zur Erkennung von Sprachbefehlen
	\item Kamera mit entsprechenden Bibliotheken zur Erkennung von Bar- und QR-Codes
\end{itemize}
Die, für die oben beschriebene \acf{AuthN} übliche Eingabemethode, die textbasierte Eingabe über eine entsprechende Tastatur ist über die \ac{VR}-Brille ohne zusätzliche Hardware nicht möglich. Zusätzliche Hardware wäre zu dem umständlich und würde die Bedienung des Gerätes erschweren. Die Usability ist bei dieser Eingabemethode nicht gegeben.\\

Die Eingabe der Anmeldedaten muss daher über andere Eingabemethoden stattfinden und wird in 2 Teile unterteilt:
\begin{enumerate}
	\item Eingabe/Auswahl des Benutzernamens
	\item Eingabe des Passworts
\end{enumerate}

Der Benutzername ist - im Gegensatz - zum Passwort zumindest gegenüber den anderen Mitarbeitern, die Zugriff auf die Brille haben, kein Geheimnis und kann Bekannt sein.\\
Die Eingabe des Benutzernamens über ein Sprachkommando gestaltet sich schwierig und als nicht effektiv. Im Rahmen dieser Arbeit wurde ein Test (TODO: Kapitelrefernzierung auf Kapitel mit Test der Spracherkennung) durchgeführt, der die Spracherkennung testete. Dies funktionierte bei vordefinierten Sprachbefehlen und bei wenig Störgeräuschen zufriedenstellend. Für die Eingabe von Benutzernamen ist dies jedoch nicht geeignet, da die Anmeldung sowohl in ruhigen Umgebungen, als auch lauten Filialen schnell funktionieren muss. Darüber hinaus kann die Erkennung von Eigennamen, die eventuell durch verschiedene Sprachen geprägt sind, nicht zuverlässig garantiert werden.\\
Die Entscheidung fiel daher auf eine Liste, die beim Starten der App vom Server abgerufen wird und auf der LogIn-Seite der App angezeigt wird. Der Server liefert eine Liste mit Benutzern zurück, die für die Brille zugelassen sind (TODO: Kapitelreferenzierung - Rechte). Der Benutzer wählt über die Knöpfe an der Brille den Benutzernamen aus der Liste aus und wird anschließend zur Eingabe des gültigen Passworts aufgefordert.\\
Dies garantiert eine, nach den Möglichkeiten der Vuzix M100 gegebenen, zuverlässige und schnelle Anmeldung. Diese Anmeldemethode ist verständlicherweise nur für eine geringe Anzahl an Benutzern (<30) effizient, jedoch wird davon ausgegangen, dass in einem Supermarkt in der Regel nicht mehr als 20 bis 30 Angestellte mit der Warenannahme/-einräumung beauftragt werden.\\

Auch bei der Passworteingabe gibt es ähnliche Probleme, jedoch muss hier darauf geachtet werden, dass Passwörter ausschließlich dem jeweiligen Benutzer (und eventuell dem Systemadministrator) bekannt sein dürfen \bzw nur im Besitz des Benutzers liegen dürfen. Eine Auswahl aus einer Liste und die Eingabe per Spracherkennung sind somit nicht nur aus Sicht der Bedienung, sondern vor Allem aus Sicherheitsgründen nicht praktikabel.\\
Ein Passwort, das auf Gesten basiert, ist aufgrund der geringen Anzahl an Variationen und möglichen Kombinationen ebenfalls nicht sicher.\\
Die Passworteingabe muss daher über die vierte Eingabemöglichkeit getätigt werden: die Eingabe über Barcodes/QR-Codes mit Hilfe der Kamera.\\

Sobald der Benutzer seinen Benutzernamen aus der Liste ausgewählt hat, wird die Kamera, sowie eine Bibliothek zur Erkennung von QR-Codes gestartet. Der Benutzer scannt seinen persönlichen QR-Code, der in einen String mit einer Länge von 64 Zeichen umgewandelt wird. Diese Daten werden an den Server weitergeleitet, der die Anmeldung schließlich bestätigt (bei korrekter Kombination) oder widerruft (bei ungültigen Login-Daten). Bei korrekter Authentifizierung, gibt der Server einen \ac{JWT} zurück, welcher bei einer Anfrage an den Server mitgeschickt werden muss und auf Korrektheit überprüft wird. Bei einem widerrufenen Login erhält die App ausschließlich eine Fehlermeldung, weitere Anfragen werden aufgrund des fehlenden \ac{JWT} nicht ausgeführt.\\

Der QR-Code kann mit Hilfe der Weboberfläche generiert werden oder es kann ein bestehender Code aktiviert werden (TODO: Kapitelreferenz SMAR Web Administration). Der QR-Code kann durch den Benutzer aufbewahrt werden, sollte der Code in unbefugte Hände gelangen, kann ein neuer Code generiert werden. Außerdem ist es möglich vorhandene Codes, wie \zB ein Code auf der persönlichen Firmen-Zugangskarte des Benutzers, zu verwenden.\\

Die benötigte Sicherheit und das Effiziente Anmelden an die Anwendung ist mit dieser Lösung gewährleistet.

\subsection{\acs{AuthN} gegenüber dem Server}
\label{cha:authn_server}
Im letzten Kapitel wurde beschrieben, wie sichergestellt wird, dass sich nur bekannte und authorisierte Benutzer anmelden können. Das dies nicht unbedingt ausreichend ist, verdeutlicht das folgende Szenario:
\begin{itemize}
	\item Ein Angestellter einer Filiale, der mit der Warenannahme beauftragt ist und somit für die Nutzung der Brille freigeschaltet ist, lässt seinen Firmenausweis beim Einräumen in einem öffentlichen Bereich liegen. Auf dem Firmenausweis sind sowohl der vollständige Name, als auch der QR-Code, der für die Authentifizierung genutzt wird, aufgedruckt. Ein Angreifer, der Zugriff auf das System bekommen möchte, entdeckt dies und fotografiert den Firmenausweis ab.
\end{itemize}
Im oben dargestellten Szenario sind die persönlichen Benutzerdaten - ohne Wissen des Opfers - gestohlen worden. Der Angreifer hat zwar keinen Zugriff auf eine im Markt vorhandene \ac{VR}-Brille, aber eventuell ist er im Besitz der App und hat diese auf einem eigenen Android-Gerät installiert. Ist das Firmennetzwerk zusätzlich noch schlecht abgesichert, \zB durch Nutzung des \ac{WEP} Verschlüsselungsprotokolls, so kann sich der Angreifer mit seinem eigenen Android-Gerät und über die Anmeldedaten des Angestellten auf dem Server authentifizieren.\\
Er kann den Warenbestand nun entsprechend manipulieren und der Filiale schaden zufügen.\\

Um dies zu verhindern, wurde eine Zwei-Faktor-Authentifizierung in die Anwendung integriert. Somit muss sich nicht nur der Benutzer, sondern ebenfalls die \ac{VR}-Brille \bzw das benutzte Gerät gegenüber dem Server authentifizieren.\\

Die App liest dazu beim Starten der Anwendung die \ac{MAC}-Adresse des Gerätes aus und speichert diese in der für den Login zuständigen Klasse ab. Sobald sich der Benutzer anmeldet (seinen Benutzernamen ausgewählt hat und den persönlichen QR-Code eingescannt hat), wird die \ac{MAC}-Adresse an die Login-Daten angehängt und eine Anfrage (Ausführen der "authenticate"-Anwendung der REST Api\footnote{s. Kapitel TODO: Kapitelreferenz auf REST Api-Erklärung}) an den Server mit allen Daten (\ac{MAC}-Adresse, Benutzer, Passwort) geschickt. Ist die \ac{MAC}-Adresse gültig, liefert der Server den \ac{JWT}, ansonsten gibt es eine Fehlermeldungen und alle weiteren Anfragen werden abgelehnt.\\

Eine \ac{MAC}-Adresse und somit das dazugehörige Gerät werden durch einen Eintrag in der Datenbank registriert. Für jedes Gerät wird ein Name, sowie die \ac{MAC}-Adresse vergeben und ein Flag gesetzt, ob dieses Gerät aktuell aktiv sein soll. Das registrierte Gerät kann sich somit gegenüber dem Server erfolgreich identifizieren.

\section{\acf{AuthZ}}
Die \acl{AuthZ} beschreibt - im Gegensatz zur \acl{AuthN} - nicht die Identifikation einer Person/eines Gerätes, sondern prüft die Berechtigungen der bereits vorhandenen Identifikation. Für eine \acl{AuthZ} ist daher eine bereits erfolgreiche \acl{AuthN} erforderlich.

Bei der Benutzung der App gibt es sowohl für das Gerät, als auch für den Benutzer ausschließlich 2 Berechtigungsstufen:
\begin{itemize}
	\item Benutzer/Gerät hat die Berechtigung Daten zu lesen/bearbeiten/löschen,
	\item Benutzer/Gerät hat keine Berechtigung auf die Daten zuzugreifen.
\end{itemize}

Die \acl{AuthZ} findet daher zeitgleich zu der \acl{AuthN} statt. Der \ac{JWT} wird ausschließlich bei erfolgreicher Identifikation und Berechtigung zurückgegeben, ansonsten gibt es eine entsprechende Fehlermeldung.\\

Das Gerät autorisiert sich gegenüber dem Server über das Flag, welches bestimmt, ob das Gerät aktuell aktiv sein soll. Ist dieses Flag gesetzt, besitzt dieses Gerät (\zB \acs{VR}-Brille) volle Berechtigung und weitere Anfragen werden bei gültigem \ac{JWT} übergeben, ansonsten werden alle weiteren Anfragen abgelehnt. Diese \acl{AuthZ} macht Sinn, sollte das Gerät kurzfristig nicht in der Filiale sein. Das Gerät kann gesperrt und reaktiviert werden, ohne dass es gelöscht und anschließend wieder registriert werden muss.\\

Da ein Benutzer ebenfalls nur diese zwei Berechtigungsstufen beim Benutzen der App besitzt, wird die Anfrage ähnlich der Brille ausgeführt. Ein Benutzer besitzt in seinem Datenbank-Eintrag in der Spalte \glqq role\_device\grqq~~ entweder eine 1 (true) und wird autorisiert oder eine 0 (false) und der Server meldet einen entsprechenden Fehler zurück.\\

Weitere Berechtigungsstufen werden an dieser Stelle nicht benötigt, da ein Angestellter, der mit der Warenannahme oder -einräumung beauftragt ist, die gesamte App-Funktionalität benutzt und ein Angestellter, der keine der beiden Aufgaben benötigt, keinen Zugriff auf die Brille benötigt.
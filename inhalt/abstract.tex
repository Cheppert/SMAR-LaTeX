\chapter*{Abstract} %*-Variante sorgt dafür, das Abstract nicht im Inhaltsverzeichnis auftaucht
\sloppy
Die folgende Arbeit beschäftigt sich mit der Optimierung des \glqq Shelf-Managements\grqq~im Einzelhandel. Dabei wurde versucht, veraltete und historisch gewachsene Verfahren durch neue Prozesse, die durch \glqq Wearable-Computern\grqq~unterstützt werden, abzulösen.\\

Ziel dieser Arbeit ist das Entwickeln einer Gesamtlösung, bestehend aus einer Administrationsoberfläche zur Bestands-/Regal-/Produktverwaltung und einer Android App, die sämtliches Papier während der Warenannahme und -einräumung ablösen soll. Diese Lösung soll als \glqq Proof of Concept\grqq~verstanden werden und die nötige Flexibilität für Live-Demos in kleineren Filialen mitbringen.\\

Externe durchgeführte Studien, sowie eine im Rahmen dieses Projektes durchgeführte Umfrage bei Filialleitern einer Supermarktkette\footnote{Name aus datenschutzrechtlichen Gründen nicht genannt.} stellen die aktuellen Probleme beim Shelf-Management im Einzelhandel dar. Darüber hinaus stellen diese Informationen die Grundlage für die durchgeführte Bedarfs- und Anforderungsanalyse dar. Auf Grundlage der Anforderungen wird die Architektur entwickelt.\\

Die Konzepte der Implementierung sowie die verwendete Technik werden detailliert erklärt, um eine Weiterentwicklung und Fortführung dieses Projektes zu ermöglichen.\\

Alle definierten Ziele der ersten Priorität wurden im gegebenen Zeitrahmen erfüllt und bilden die Grundlage für einen erfolgreichen \glqq Proof of Concept\grqq . Da ein Testen der Anwendung in realem Umfeld im Rahmen dieser Arbeit nicht möglich war, können eventuell auftretende Schwächen und falsch interpretierte Anforderungen nicht herausgestellt werden. Die Anwendung benötigt daher weitere Entwicklungszeit und Analyse während der ersten Tests. Darüber hinaus fallen bereits während der Entwicklung Schwachstellen in der zur Verfügung stehenden Technik auf: das Display der Smartglass wirkt undeutlich und erschwert das Ablesen der nötigen Informationen.\\

Das Projekt ist im Rahmen der gestellten Ziele ein Erfolg und zeigt die Möglichkeiten und Chancen auf, die sich durch ein computergestütztes Shelf-Management ergeben.
\fussy
% Sperrvermerk bei Bedarf dekommentieren
%\hrule 
\chapter*{Sperrvermerk}
Die vorliegende Arbeit mit dem Titel {\itshape \titel - \untertitel} enthält vertrauliche Informationen der\\

\textbf{SAP SE\\
Dietmar-Hopp-Allee 16\\
69190 Walldorf}\\

Sie ist mit einem Sperrvermerk versehen und wird ausschließlich zu Prüfungszwecken für {\studiengang} der Dualen Hochschule Baden-Württemberg Mannheim vorgelegt.\\
Jede Einsichtnahme und Veröffentlichung – auch von Teilen der Arbeit – bedarf der vorherigen Zustimmung durch die {\firma}.


%\newpage
%
%
%\addchap*{Erklärung}
%gemäß § 5 (3) der \glqq Studien- und Prüfungsordnung DHBW Technik\grqq vom 22. September 2011.
%Ich habe die vorliegende Arbeit selbstständig verfasst und keine anderen als die angegebenen Quellen und Hilfsmittel verwendet.\\
%
%%Ich versichere hiermit, dass ich meine \arbeit~ mit dem Thema
%%\begin{quote}
%%\textit{\titel} -\textit{ \untertitel }
%%\end{quote}
%%selbständig verfasst und keine anderen als die angegebenen Quellen und Hilfsmittel benutzt habe. Die Arbeit wurde bisher keiner anderen Prüfungsbehörde vorgelegt und auch nicht veröffentlicht.\\
%
%%Mir ist bekannt, dass ich meine \arbeit zusammen mit dieser Erklärung fristgemäß nach Vergabe des Themas in doppelter Ausfertigung und gebunden im Sekretariat meines Studiengangs an der DHBW Mannheim abzugeben habe. Als Abgabetermin gilt bei postalischer Übersendung der Eingangsstempel der DHBW, also nicht der Poststempel oder der Zeitpunkt eines
%%Einwurfs in einen Briefkasten der DHBW.\\[10ex]
%
%\location, \today \\[4ex]
%\rule[-0.2cm]{5cm}{0.5pt} \\
%\textsc{\autor} \\[10ex]













